\chapter{Conclusion}

Understanding the balance of evolutionary forces shaping the origin and maintenance of genetic variation has been the core driver of population genetics \citep{lewontin_genetic_1974}.
Our ability to collect data has exponentially increased over the last few decades, moving from allozyme gels of a few samples to whole-genome data of thousands of individuals.
With this flood of data, we are poised to make huge progress on long standing evolutionary questions.
However, the traditional framework for evolutionary inference has somewhat stalled new discoveries.

Many of the issues we are facing can be mitigated with evolutionary simulations.
A great deal of progress has been made on evolutionary inference tools \citep{haller_slim_2019,kelleher_efficient_2016, haller_tree-sequence_2019,adrion_community-maintained_2020}.
These advancements, coupled with huge increases in computational power, now allow us to use simulations to effectively infer previously intractable likelihoods,
and to explore features of the data that are not easy to model mathematically.
Indeed, over the past decade simulation-based inference has gained immense popularity \citep{schrider_shic_2016,torres_human_2018,caldas_inference_2022,chan_likelihood-free_2018,korfmann_simultaneous_2023}.


