\chapter{Introduction}

\section{Decoding evolution: a lay-friendly prelude}

Evolution can occur over incredibly large spatial scales over the course of thousands to millions of years.
This makes Evolutionary biology unlike other fields within Biology in that hypotheses can not always be subject to experimentation.
Thus, we are left with the task of putting together what happened in the past based on information we have today.
To do so we need a sizable amount of data, models and statistical and computational tools.

One way to make inferences about past evolutionary processes is by studying genetic variation within and between species.
Evolutionary processes, such as demographic processes, natural selection and mutation, impact the genetic variation in many ways.
For example, we expect the amount of variation within a species to be proportional to the mutation rate (\ie the higher the mutation rate, the more variation a population will harbor).
On the other hand, finite populations lose variation every generation due to genetic drift (\ie sampling error), 
and the magnitude of this loss is proportional to the population size.

We can write models in form of math equations that describe how some of these evolutionary processes impact genetic variation.
For example, $\pi$ (a measure of genetic variation) depends on the size of the population ($N$) and the mutation rate ($\mu$), such that $\pi \sim 4N\mu$.
If we knew the mutation rate, we could rearrange the equation to estimate the size of a population $N = \frac{\pi}{4\mu}$ from genetic data!
To arrive at this simple equation, we make the assumption that an individual is equally likely to mate with any other individual in the population (\ie panmixia).
It is easy to see how this may not always hold:
for example, a tree is much more likely to mate with its neighbor than a distant tree across the forest (because pollen cannot travel too far).

Adding biological realism to our math equation would prove to be complicated.
Alternatively, we can more easily simulate this process with a computer:
we can create virtual individuals and tell a computer how they choose their mates, pass on genetic material to their offspring, and change over time.
If we were to run virtual experiments like this many times,
it would be possible to map the relationship between genetic variation ($\pi$) and our parameters of interest (population size $N$ and mutation rate $\mu$).
This new map would be closer to the reality than our simplified mathematical model above,
but this approach is not a panacea, as simulations have a high computational cost (\ie they can take a long time to run).

For my dissertation, I have explored how we can incorporate computer simulations to help us understand evolution.
To this end, we need computational tools that are fast and reproducible (so that other researchers can use them).
So, in my second chapter, I discuss my contributions to tools that make these simulations run faster and to make them more reproducible.
Next, I sought to answer a long standing biological question: to what extent have humans and other apes been shaped by natural selection?
In my third chapter, I use simulations to tackle this question and show that both positive and negative selection are necessary to explain the genetic data obtained from multiple individuals for all great ape species.
Lastly, I wanted to build a proper statistical method that uses simulations to infer evolutionary processes from genetic data.
In my fourth chapter, I present a new method for evolutionary inference that uses genealogical trees along chromosomes
(instead of tables with samples and mutations).
These trees capture the genetic data perfectly, but they are much more scalable (use less space) and
they organize the data in a way that could be more conducive to extracting relevant evolutionary information.

\section{The current landscape in inference of evolutionary processes from genetic data}

% What are the major questions in evolutionary genetics?
Many questions in evolutionary biology are of a historical nature,
because evolution is restrict in space and time and it cannot be replicated \citep{losos_evolutionary_2009, cleland_methodological_2002}.
An avenue for investigating past evolutionary events and processes lies in genetic data.
As evolutionary processes leave footprints on the genetic composition of populations,
the possibility arises to invert this relationship,
that is to use genetic data for the inference of past events.
Indeed, much evolutionary genetics is tasked with trying to infer past evolutionary events and processes from genetic data.
Among the major questions explored in the field are: 
Is it possible to find regions of the genome that were affected by natural selection?
Can we infer the movement of individuals and changes in population sizes over time?
What are the relative contributions of different evolutionary processes, such as demography and selection, in shaping patterns of genetic variation? 

\subsection{The impact of evolutionary processes on genetic variation}

% What kinds of signatures do evol processes leave on genetic data?
To understand how it is possible to answer these questions with genetic data,
it is necessary to map how different evolutionary processes affect genetic variation.
At each position along a chromosome, there is a tree that describes how sampled individuals are related to each other.
Due to linkage, neighboring trees are more like each other than distant trees.
These trees or genealogies can fully encode the outcomes of evolutionary processes.
Therefore, it is usually more intuitive to think about how these processes impact tree shapes.
The trees, along with mutations, can then be used to understand genetic variation itself.

Demographic processes directly impact the underlying genealogies because of the inverse relationship between population size and coalescence rate(coalescences are the merging of lineages backwards to a most recent common ancestor) \citep{wakely_coalescent_2016}.
For example, for a population that goes through a bottleneck (\ie a sudden decrease in population size),
the coalescences are expected to be concentrated during the bottleneck period.
On the other hand, for population that undergoes a size expansion,
most of the coalescences will happen in the period preceding the expansion.
These underlying trees yield different patterns of genetic variation in the extant population.
One summary of genetic data that can be used to distinguish between these tree shapes and, in turn, between possible demographies is the site frequency spectrum (SFS).
The SFS is the distribution of allele frequencies across sites in some sampled genomes.
(For the sake of simplicity, I will focus on the unfolded SFS, where the frequency of the derived alleles have been measured; derived allele refers to the allele that is new to a population).
A genealogy that has coalescences concetrated near the present (due to a recent bottleneck) will yield a SFS that is skewed towards the high frequency of derived alleles.
Conversely, a genealogy with coalescences happening earlier on will yield a SFS with an excess of low frequency derived alleles.

% Natural selection and the ways it impacts genetic variation
Natural selection can also impact genealogies and genetic variation.
When a new (sufficiently) beneficial mutation arises in a population it rapidly increases in frequency, ultimately reaching fixation.
This leads to many of the coalescences to occur during the fixation trajectory.
If we were to sample the population right after fixation,
the SFS would be skewed towards the high frequencies \citep{kim_allele_2006}, similar to the effect of a population bottleneck.
A strongly deleterious mutation, on the other hand, is rapidly purged from the population.
Therefore, the shape of the underlying tree is not affected, though the tree will be shorter \citep{williamson_genealogy_2002, barton_effect_2004}.

Both beneficial and deleterious mutations lead to loss of genetic variation at the selected site (either due to fixation or purging),
but nearby sites are also affected due to linkage.
Linked mutations are carried along as a beneficial mutation increases in frequency, in a process called selective sweep \citep{kaplan_hitchhiking_1989, maynard_smith_hitch-hiking_1974, coop_patterns_2012}.
The further away from the focal site, the more likely it is for recombination to happen, allowing mutations to escape this "sweep".
A deleterious mutation can also affect linked mutations in a similar way, and this process is called background selection \citep{charlesworth_effect_1993}.
Importantly, most of the models for selection assume strong selection,
because it is complicated to deal with the interactions between multiple selected mutations in linkage disequilibrium \citep{kim_joint_2000}.

\subsection{Challenges in evolutionary inference}

Although we have a good understanding of how different evolutionary processes impact variation,
a few issues remain in the way of inversing this relationship to infer these processes from genetic data.
First, different processes can impact genetic variation in similar ways, making it difficult to determine the specific processes that are consistent with data.
Second, there is a need for modeling of interactions between processes.
Third, many models do not properly incorporate linkage information and treat whole genome data as unlinked single loci.

Summaries of genetic data may not contain enough information to distinguish between evolutionary scenarios.
For example, both positive and negative selection remove variation surrounding the selected site.
Because the width of the effects of sweeps and background selection depends on the recombination rate,
both these models predict relationships between (i) recombination rate and genetic diversity and (ii) density of functional sites and genetic diversity.
Indeed, in many species of plants and animals it has been observed that regions of high recombination harbor more genetic variation \citep{corbett-detig_natural_2015}.
Disentangling the relative contributions of sweeps and background selection to these patterns, however, is complicated \citep{leffler_revisiting_2012, andolfatto_adaptive_2001}.
One alternative lies in going beyond single species metrics of variation, and to instead compare multiple species.
Selection also directly impacts divergence between species: whereas positive selection increases fixation rates, 
negative selection decreases the rate of substitutions \citep{andolfatto_hitchhiking_2007,macpherson_genomewide_2007,cai_pervasive_2009} .
It may also be possible to disentangle sweeps and background selection by aggregating multiple different measures of variation \citep{schrider_background_2020}. % cite schrider

Individually dissecting and modelling the effects of different processes on genetic variation is not enough.
In the early days of evolutionary genetics, it was possible to survey just a few genetic markers, usually microsatellites.
These markers are short tandem repeats (1-6bp) that are abundant in the genomes of many taxa.
Microsatellites have a high mutation rate and are thought to be mostly neutral \citep{field_long_1997}.
This allowed researchers to use these markers for understanding demographic processes almost independently of other processes, such as natural selection.
There have been incredible advancements in our ability to obtain whole-genome data,
and now it is feasible to sequence thousands of individuals for multiple species at once.
More data opens up the opportunities to investigate more complex questions (\ie beyond the traditional studies of population structure),
such as understanding constraint along genomes, recombination rate variation, among others.
However, with whole-genome data the assumptions of neutrality, for example, are much less likely to hold.
So there is a need for models that account for complex interactions between processes, which is not trivial to do.
One interesting (and perhaps inintuitive) way that processes can interact is that demographic processes can exacerbate the effects of background selection \citep{torres_human_2018, torres_temporal_2020}.
Inferences of background selection that do not take into account demography can lead to biased estimates of the rate of deleterious mutations and their fitness effects.
Similarly, there is a rich literature describing how the effects of both positive and negative selection can bias demographic inferences \citep{schrider_effects_2016, ewing_consequences_2016, johri_impact_2021}.

Taking full advantage of linkage information is complicated,
and many current methods treat whole-genome data as a collection of unlinked loci.
Many demographic inference methods based on allele frequency data (\ie the SFS) do not make use of information contained in patterns of linkage between sites.
More recently, methods have started to incorporate patterns of haplotype diversity for inference by modeling the distribution of regions of identity by descent (IBD).

\subsection{The promise of simulation-based evolutionary inference}
% Data is not a limitation anymore

% Computational and simulation advancements

% This makes simulation-based inference attractive to solve big problems in evo gen. What is simulation-based inference and how can it be applied to evogen?

% To make simulation-based inference possible, we need to maintain and develop scalable, tested and documented tools in pop gen (Chapter 1)

% In chapter 2, we present a large scale simulation study investigating the role of positive and negative selection in shaping genetic variation.

% In chapter 3, we develop a supervised machine learning method to infer evol params from data, exploring new facet of genetic variation.

