\chapter{Introduction}

Evolutionary biology is unlike other fields within biology in that hypotheses can not always be subject to experimentation,
because evolution can occur over incredibly large spatial scales over the course of thousands to millions of years.
Thus, we are left with the task of putting together what happened in the past based on information we have today.
To do so we need a sizable amount of data, models and statistical tools.

One way to make inference about past evolutionary processes is by studying genetic variation within and between species.
Evolutionary processes, such as demographic processes, natural selection and mutation, impact the genetic variation in many ways.
For example, with higher mutation rates, we expect to see more variation within a species.
On the other hand, in finite populations genetic variation is lost due to sampling (or genetic drift).

We can write models in form of math equations that describe how some of these evolutionary processes impact genetic variation.
For example, $\pi$ (a metric of genetic variation) depends on the size of the population ($N$) and the mutation rate ($\mu$), such that $\pi \sim 4N\mu$.
If we knew the mutation rate, we could rearrange the equation to estimate the size of a population $N = \frac{\pi}{4\mu}$!
The simple equation, however, does not hold in real world conditions, because there are assumptions behind the equality that are unrealistic.
For example, the equation assumes that an individual is equally likely to mate with any other individual.
Although this simplifying assumption could hold for some species, it is clear that it would not hold for humans, for example.
People are much more likely to mate with someone that lives where they live.

Adding this geographical realism to our math equation would prove to be quite complicated.
Alternatively, we can more easily simulate this process with a computer:
that is, we can create virtual individuals and tell a computer how they mate and change over time.
If we were to do this many times, it would be possible to obtain a map from 

% What are the major questions in evolutionary genetics?

% What challenges are in the way of answering these questions?

% Data is not a limitation anymore

% Computational and simulation advancements

% This makes simulation-based inference attractive to solve big problems in evo gen. What is simulation-based inference and how can it be applied to evogen?

% To make simulation-based inference possible, we need to maintain and develop scalable, tested and documented tools in pop gen (Chapter 1)

% In chapter 2, we present a large scale simulation study investigating the role of positive and negative selection in shaping genetic variation.

% In chapter 3, we develop a supervised machine learning method to infer evol params from data, exploring new facet of genetic variation.


\section{Chapter One Section One}
\subsection{Chapter one section one sub-section one}

\chapter{Inferring evolutionary parameters from whole-genome genealogies using machine learning}

% Population genetic inference has relied on summaries of genotype data

% A tree structure can make evolutionary events and processes more readily interpretable. And now these can be estimated.

% Still, we rely on summaries of these trees to make inference.

% Developments in machine learning now allow for supervised learning based off of graphs.

% Here, we apply networks that take the graph structure into account to infer evolutionary parameters.

