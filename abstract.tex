\abstract{
Evolutionary processes impact patterns of genetic variation,
so there is an opportunity to reverse this relationship and use genetic data to learn about past evolutionary events.
Traditional evolutionary inference from genetic data is plagued by a few issues:
(i) different processes can impact a particular feature in similar ways, 
making it difficult to disentangle them;
(ii) there is a growing need for modeling interactions between processes; and
(iii) many models do not make full use of genomic data and instead assume that loci are unlinked.

Simulation-based evolutionary inference can help alleviate many of these issues.
It is now possible to simulate complex evolutionary scenarios, 
and these can be used to approximate analytically intractable likelihoods,
for example by using supervised machine learning.
The major downsides to using simulations is the computational cost,
but recent advancements both in hardware and software have lessened this cost.
In this dissertation,
I pushed the boundaries on how simulation-based inference can be applied in evolutionary genetics.

To mitigate the costs associated with simulations,
I made a few contributions to the \tskit ecosystem of evolutionary simulation tools.
First, I developed a way to partially parallelize the simulation of multiple populations to make inference using multi-population genomic datasets more feasible.
Second, I helped create standards for reproducible simulations with natural selection within the \stdpopsim consortium.
I implemented the ability to simulate using previously published distribution of fitness effects (DFEs) and to simulate selective sweeps.
I demonstrate the utility of this tool by tackling the long-standing question of whether the power to detect sweeps varies along realistic chromosomes.

Next, I used simulations to better understand the behavior of a complex multi-population model.
Species can be thought as semi-independent realizations of the same (or very similar) evolutionary process.
Thus, by looking at multiple species at once it may be possible to better disentangle the processes that shape variation along genomes.
Using simulations, I show that positive selection is necessary to explain the genetic data obtained from multiple great ape species.
Further, I lay down a framework for leveraging multi-species information to better understand the effects of different processes on a group's evolutionary history.

Lastly, I present a new method that uses whole-genome genealogies for evolutionary inference.
This data structure efficiently and sufficiently encodes evolutionary processes.
I develop a machine-learning framework, tsNN, that takes whole-genome genealogies as inputs
and is flexible enough to perform tasks at different scales (\eg inferring mutation times, demographic parameters, etc.).
I demonstrate that tsNN can learn to predict mutation times accurately, outperforming current likelihood-based methods.
tsNN, represents an important step in genealogy-based evolutionary inference, 
but there still much work to be done in applying deep learning to gain new insights into past evolutionary events.

}
